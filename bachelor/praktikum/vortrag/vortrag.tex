\documentclass{beamer}
%\documentclass[handout]{beamer}

\usepackage[german]{babel}
\usepackage[latin1]{inputenc}
\usepackage{helvet}
\usepackage[T1]{fontenc}
\usepackage{ifpdf}
\usepackage{multimedia}
\usepackage[sans]{dsfont}
\usepackage{pgfpages}
\usepackage{xspace}
\usepackage{threeparttable}
\usepackage{%
  array,
  rotating,
  booktabs,
  colortbl,
  multirow
}


\mode<presentation> {
  \usetheme{LME}
}

\mode<handout> {
  \setbeameroption{show notes}
% \pgfpagesuselayout{resize to}[a4paper,border shrink=7mm,landscape]
  \pgfpagesuselayout{2 on 1}[a4paper,border shrink=7mm]
% \pgfpagesuselayout{4 on 1}[a4paper,border shrink=7mm,landscape]
}


\title{\vspace*{0.3cm} Visualisierung von Spektrogrammen}
% \subtitle {\footnotesize This is my subtitle} 
\subject{Visualisierung von Spektrogrammen}
\author[Dominik Neumann, Felix Lugauer] {Dominik Neumann, Felix Lugauer}
\date{12.03.2010}

\institute[Lehrstuhl f\"ur Informatik 5] {
  Lehrstuhl f{\"u}r Informatik 5 (Mustererkennung)\\
  Universit{\"a}t Erlangen-N{\"u}rnberg\\
}



\newcolumntype{R}[1]{%
  >{\begin{turn}{90}\begin{minipage}{#1}\raggedright\hspace{0pt}}l%
  <{\end{minipage}\end{turn}}%
}


% Folgendes sollte gelöscht werden, wenn man nicht am Anfang jedes
% Unterabschnitts die nochmal sehen möchte.
\AtBeginSection[] {
  \begin{frame}<beamer>
    \frametitle{Overview}
    \tableofcontents[currentsection] %,currentsubsection]
  \end{frame}
}

% Falls Aufzählungen immer schrittweise gezeigt werden sollen, kann
% folgendes Kommando benutzt werden:
%\beamerdefaultoverlayspecification{<+->}


%%%%%%%%%%%%%%%%%%%%%%%%%%%%%%%%%%%%%%%%%%%%%%%%%%%%%%%%%%%%%%%%%%%%%%%%%%%%%%%%%%%
% Eigene Definitionen

\newenvironment<>{codeblock}[1]%
{\setbeamertemplate{blocks}[rounded][shadow=true]%
   \begin{exampleblock}{#1}}
{\end{exampleblock}%
   \setbeamertemplate{blocks}[default]}

\definecolor{bgblue}{rgb}{0.04,0.39,0.53}  % 10  99  136  %HEX: A  63  87
\definecolor{lmeblue}{rgb}{0.0,0.0,0.6}
\definecolor{blue2}{rgb}{0.0,0.2,0.9}
\definecolor{bgred}{rgb}{0.8,0.2,0.2}    % 204  51  51    %HEX CC 33 33
\definecolor{c1}{rgb}{1.0,1.0,0.0}
\definecolor{c2}{rgb}{1.0,0.5,0.0}
\definecolor{c3}{rgb}{1.0,0.0,0.0}
\definecolor{c4}{rgb}{1.0,0.0,0.5}
\definecolor{c5}{rgb}{1.0,0.0,1.0}
\definecolor{c6}{rgb}{0.5,0.0,1.0}
\definecolor{c7}{rgb}{0.0,0.0,1.0}
\definecolor{c8}{rgb}{0.0,0.5,1.0}
\definecolor{c9}{rgb}{0.0,1.0,1.0}
\definecolor{c10}{rgb}{0.0,1.0,0.5}
\definecolor{c11}{rgb}{0.0,1.0,0.0}
\definecolor{c12}{rgb}{0.5,1.0,0.0}
\definecolor{c13}{rgb}{1.0,1.0,1.0}
\definecolor{gold}{rgb}{.99,1,.9}%
\def\b{\color{LMEblue}\bf}
\def\bb{\color{blue2}\bf}
\def\r{\color{bgred}\bf}
\def\yellowbox{\colorbox[rgb]{1,1,0.8}}
\let\Otemize =\itemize
\let\Onumerate =\enumerate
% Zero the vertical spacing parameters
\def\Nospacing{\itemsep=0pt\topsep=0pt\partopsep=0pt\parskip=0pt\parsep=0pt\left
margin 2.5em}
\def\Sspacing{\itemsep=0pt\topsep=0pt\partopsep=0pt\parskip=0pt\parsep=0pt\leftm
argin 2.5em}
\def\ispacing{\itemsep=7pt\topsep=5pt\partopsep=5pt\parskip=5pt\parsep=5pt\leftm
argin 2.5em}
\def\Mspacing{\itemsep=10pt\topsep=0pt\partopsep=0pt\parskip=0pt\parsep=0pt\left
margin=10.5cm}
% Redefine the environments in terms of the original values
%\renewenvironment{itemize}{\Otemize\ispacing}{\endlist}
\newenvironment{Itemize}{\Otemize\Nospacing}{\endlist}
\newenvironment{Stemize}{\Otemize\Sspacing\small}{\endlist}
\newenvironment{Enumerate}{\Onumerate\Nospacing}{\endlist}
\newenvironment{Mtemize}{\Otemize\Mspacing}{\endlist}
\def\bitem{\item[$\bullet$]}
\def\forts{{\tiny (Forts.)}}
\def\stopp#1{\pause\vspace*{#1}}
\def\spread{\vspace*{\fill}}
\def\pspread{\pause\spread}



\begin{document}

\input{commands.own}

\setbeamertemplate{background canvas}[LME title]
\setbeamertemplate{navigation symbols}{}
\begin{frame}[plain]
  \titlepage
\end{frame}

\setbeamertemplate{background canvas}[default]

\mode<beamer> {
  \setbeamertemplate{navigation symbols}{
    \insertslidenavigationsymbol
    \insertsubsectionnavigationsymbol
    \insertbackfindforwardnavigationsymbol
  }
}


% Verwendung des pdf-Formats bei ps-Bildern, wenn mit pdflatex bersetzt wird
\ifpdf
  \def\ps{pdf}  \def\psdir{./eps_c}
  \def\jpg{jpg} \def\jpgdir{./jpg}
  \def\fig{pdf} \def\figdir{./fig_c}
  \def\texfig{pdf} \def\texfigdir{./texfig_c}
\else
  \def\ps{eps}  \def\psdir{./eps}
  \def\jpg{eps} \def\jpgdir{./jpg_c}
  \def\fig{eps} \def\figdir{./fig_c}
  \def\texfig{pstex} \def\texfigdir{./texfig_c}
\fi


\begin{frame}
  \frametitle{Overview}
  \tableofcontents
  % Die Option [pausesections] könnte ntzlich sein.
\end{frame}





%%%%%%%%%%%%%%%%%%%%%%%%%%%%%%%%%%%%%%%%%%%%%%%%%%%%%%%%%%%%%%%%%%%%%%%%%
% % % % % % % % % % % % % % % % % % % % % % % % % % % % % % % % % % % % %

\section{Motivation}

\begin{frame}
	\frametitle{Aufgabenstellung}
	\structure{Paket 3:} Visualisierungskomponenten f"ur Funktionen $f(t) \in \mathds{R}^2$\\
		\begin{itemize}
			\item Komponente zur Darstellung des Spektrogramms von Audiostreams
			\item Komponente zur Darstellung des Spektrogramms von Audiodateien\\[.5cm]
		\end{itemize}
		\pause
	\structure{}\textit{Und so k"onnte es aussehen:}\\
	\begin{center}
		\includegraphics[width=\linewidth/2]{\jpgdir/wavesurfer.\jpg}
	\end{center}
\end{frame}

\begin{frame}
	\frametitle{Spektrogramme}
	\structure{Definition}
		\begin{quote}
			Ein Spektrogramm ist die Darstellung des zeitlichen Verlaufs des Spektrums eines Signals
		\end{quote}
		\begin{itemize}
			\item x-Achse: Zeit
			\item y-Achse: Frequenz
			\item Intensit"at der Bildpunkte: Amplitude der jeweiligen Frequenz im Spektrum
		\end{itemize}
		\begin{center}
			\includegraphics[width=\linewidth/2]{\jpgdir/spectrogram.\jpg}
		\end{center}
\end{frame}



%%%%%%%%%%%%%%%%%%%%%%%%%%%%%%%%%%%%%%%%%%%%%%%%%%%%%%%%%%%%%%%%%%%%%%%%%%%%%%%%

\section{Vorstellung der Komponenten}

\begin{frame}
	\frametitle{Klassenhierarchie}
	\structure{}
		\resizebox{\linewidth}{!}
			{\alt<2>{
				\input{\texfigdir/klassenhierarchie.pstex_t}
				}{
				\input{\texfigdir/klassenhierarchie_ohne_.pstex_t}
				}
			}
\end{frame}

\begin{frame}
	\frametitle{Klassenhierarchie}
%	\structure{wieviel code? einarbeitung? klassenhierarchie erst am ende entstanden; usw}
		Relevante Klassen und Interfaces\\[.3cm]
		\texttt{package visual}
		\begin{itemize}	
			\item \structure{\texttt{interface Visualizer}}\\Synchronisation mit anderen Komponenten
			\item \structure{\texttt{interface FrameChanger}}\\Kommunikation mit Komponenten auf Frame-Ebene
			\item \structure{\texttt{abstract class VisualSpectrogram}}
			\item \structure{\texttt{class VisualSpectrogramAudioStream}}\\Visualisierung von Spektrogrammen von Audio-Streams
			\item \structure{\texttt{class VisualSpectrogramAudioFile}}\\Visualisierung von Spektrogrammen von Audio-Dateien
		\end{itemize}
\end{frame}

\begin{frame}
	\frametitle{Eingliederung in das vorhandene Stream-Konzept}
	\structure{}
	\begin{center}
		\resizebox{\linewidth}{!}
			{\alt<2>{
				\input{\texfigdir/audiostream.pstex_t}
			}{
				\input{\texfigdir/streams.pstex_t}}
			}
	\end{center}
	\pause
	Visualisierer f"ur Spektrogramme von \structure{Audio-Streams} lassen sich nahtlos in das bereits bestehende Stream-Konzept eingliedern
\end{frame}

\begin{frame}
	\frametitle{Problematik bei Audiodateien}
	\structure{}
		Eingliederung eines Audiodatei-Visualisierers in das vorhandenes Stream-Konzept problematisch\\[.3cm]
		\structure{Anforderungen:}
		\begin{itemize}
			\item M"oglichkeit zur Anzeige beliebiger Teile der Audiodatei (von - bis)
			\item Nutzung unterschiedlicher Fensterfunktionen
			\item Unabh"angigkeit von anderen Visualisierungskomponenten
			\item ...
		\end{itemize}
		\pause
		\textit{... und das alles zur Laufzeit}
\end{frame}

\begin{frame}
	\frametitle{Verwendung des vorhandenen Stream-Konzepts}
	\structure{Realisierung:} Die Komponente selbst "ubernimmt einen Teil des Stream-Konzepts\\[.3cm]
		\resizebox{\linewidth}{!}{\input{\texfigdir/audiofiles.pstex_t}}
\end{frame}

\begin{frame}
	\frametitle{Funktionsumfang}
	\structure{Features der Visualisierungskomponenten}\\[.3cm]
		\begin{itemize}
			\item Visualisierung von Funktionen $f(t) \in \mathds{R}^2$\\[.1cm]
			\item Logarithmische Skala\\[.1cm]
			\item Regulierung von Helligkeit und Kontrast \\[.1cm]
			\item M"oglichkeit zur Kolorierung (Hitzeskala)\\[.1cm]
			\item Farbige Bereichsmarkierungen\\[.1cm]
			\item ...\\[.1cm]
		\end{itemize}
\end{frame}

\begin{frame}
	\frametitle{Funktionsumfang \forts}
	\structure{Weitere Features}\\[.3cm]
		\begin{itemize}
			\item Anpassung der Fensterfunktion zur Laufzeit
			\item Export des Spektrogramms als Bilddatei
			\item Zusatzinformation "uber das Spektrogramm darstellen
			\item Optimierung: "Uberfl"ussige Berechnungen vermeiden
			\item Rechenintensives von eigenen Threads ausf"uhren lassen
			\item Ausf"uhrliche Code-Dokumentation mittels JavaDoc
			\item ...
		\end{itemize}
\end{frame}

\begin{frame}
	\frametitle{Demoapplikation Audiostream}
	\structure{}Implementierung einer GUI zur Demonstration des Funktionsumfangs\\[.5cm]
	\begin{center}
		\includegraphics[width=0.8\linewidth]{\jpgdir/stream.\jpg}\\[.2cm]
	\end{center}
	\structure{\texttt{visual.DemoSpectrogramStream}}
\end{frame}

\begin{frame}
	\frametitle{Demoapplikation Audiodatei}
	\structure{}Implementierung einer GUI zur Demonstration des Funktionsumfangs\\[.5cm]
	\begin{center}
		\includegraphics[width=0.8\linewidth]{\jpgdir/file.\jpg}\\[.2cm]
	\end{center}
	\structure{\texttt{visual.DemoSpectrogramFile}}
\end{frame}

%%%%%%%%%%%%%%%%%%%%%%%%%%%%%%%%%%%%%%%%%%%%%%%%%%%%%%%%%%%%%%%%%%%%%%%%%%%%%%%%

\section{Details zur Implementierung}

\begin{frame}
	\frametitle{Einsparung von unn"otigen Berechnungen}
	\structure{Ziel:} Wiederverwendung von bereits berechneten Spektrogrammdaten\\[.3cm]
	Realisierung: 
		\begin{itemize}
			\item zun"achst werden alle Daten zur Anzeige des aktuellen Spektrogramms in einem Puffer gespeichert\\[.2cm]
			\item bei "Anderung des anzuzeigenden Bereichs:
				\begin{itemize}
					\item falls neuer Bereich innerhalb des alten Bereichs liegt: benutze Werte aus Puffer, \structure{keine Neuberechnung} 
					\item sonst: Berechnung der neuen Werte n"otig\\[.2cm]
				\end{itemize}
			\item bei "Anderung der Fensterfunktion: Neuberechnung zwingend erforderlich
		\end{itemize}
\end{frame}

\begin{frame}
	\frametitle{Optimierte Darstellung der Spektrogrammdaten}
	\structure{Ziel: Ber"ucksichtigung aller Daten durch Mittelwertbildung}\\[.2cm]	
		\begin{itemize}
			\item Einpassung der Spektrogrammdaten aus dem Puffer auf die gew"unschte Bildgr"o"se
			\item Entscheidendes Verh"altnis \texttt{ $ ratio = \dfrac{bufferLength}{imageWidth} $}
			\begin{itemize}
				\item $ ratio < 1.0 $ mehrere Pixel bekommen den gleichen Wert
				\item $ ratio > 1.0 $ mehrere Pufferwerte kommen auf einen Pixel\\[.5cm]
			\end{itemize}
			\item Optisch \textit{angenehmeres} Aussehen, Gl"attung von Rauschen
		\end{itemize}
\end{frame}

\begin{frame}
	\frametitle{Kommunikation mit Komponenten auf Frame-Ebene}
	\structure{Synchronisation der Anzeige mit Spektrum, MFCC-Diagramm, ...}\\[.3cm]
		\begin{itemize}
			\item GUI-Entwickler meldet die Komponenten der Frame-Ebene beim Spektrogramm an
			\item Realisierung durch Liste: Speichern der Referenz auf zu synchronisierende Komponenten
			\item Ereignis l"asst die Komponenten bestimmtes Frame analysieren \structure{\texttt{FrameListener.show(int startSample)}}\\[.5cm]
			\item Interfaces garantieren einheitliche Kommunikation:
				\begin{itemize}
			 		\item Spektrogramm Visualisierer: \structure{\texttt{visual.FrameChanger}} 
					\item Frame-Ebene Visualisierer: \structure{\texttt{visual.FrameListener}}	
				\end{itemize}	
		\end{itemize}
\end{frame}

\begin{frame}
	\frametitle{Farbiges Spektrogramm}
	\structure{}Hitzeskala "uber HSB-Farbmodell\\[.3cm]
		\begin{columns}
		\column{0.6\linewidth}
			\begin{itemize}
				\item HSB-Farbmodell 
					\begin{itemize}
						\item Farbton
						\item S"attigung
						\item Helligkeit\\[.2cm]
					\end{itemize}
				\item Grauskala: keine S"attigung
					\begin{itemize}
						\item Wei"s: \structure{\texttt{brightness = 1.f}}
						\item Grau: \structure{\texttt{0.f < brightness < 1.f}}
						\item Schwarz: \structure{\texttt{brightness = 0.f}}\\[.2cm]
					\end{itemize}
			\end{itemize}

		\column{0.4\linewidth}
			\begin{center}
				\includegraphics[width=\linewidth]{\jpgdir/hsb.\jpg}
			\end{center}
		\end{columns}	
		\vspace{.5cm}
		\begin{itemize}	
			\item \structure{\texttt{static int java.awt.Color.HSBtoRGB(float hue, float saturation, float brightness)}}
		\end{itemize}
\end{frame}

%%%%%%%%%%%%%%%%%%%%%%%%%%%%%%%%%%%%%%%%%%%%%%%%%%%%%%%%%%%%%%%%%%%%%%%%%%%%%%%%

\begin{frame}
	\frametitle{M"oglichkeit zur "Anderung der Fensterfunktion}
	\structure{Ziel:} Veranschaulichung von Auswirkungen der verwendeten Fensterfunktion auf das Spektrogramm\\[.3cm]
		\begin{itemize}
			\item Fenstereigenschaften
				\begin{itemize}
					\item Typ: \structure{\texttt{Hamming | Hanning | Rectangle}}
					\item Gr"o"se: Anzahl der Abtastwerte pro Fenster
					\item Shift: Sprungweite\\[.2cm]
				\end{itemize}
			\item Anpassung aller genannten Eigenschaften \structure{zur Laufzeit} m"oglich 
		\end{itemize}
	\begin{center}	
		\includegraphics[width=0.5\linewidth]{\jpgdir/vergleich_fenster.\jpg}\\[.1cm]
	\end{center}
\end{frame}

\begin{frame}
	\frametitle{Implementierung des internen Datenstroms}
		\begin{enumerate}
			\item Auslesen des gepufferten Audiosignals 
			\item Fensterung liefert Werte f"ur FFT-Berechnung
			\item FFT liefert Frames, die im internen Puffer gespeichert werden
		\end{enumerate}
		\begin{columns}
			\column{\linewidth}
			\begin{codeblock}{\footnotesize Implementierung}
				\scriptsize
				\begin{verbatim}
	public void computeAndRepaintSpectrogram() {
	
	    BufferedAudioSourceReader basr = bas.getReader(viewFromSample,viewNSamples);	
	    window = Window.create(basr, windowType+","+windowLength+","+windowShift);
	    FrameSource spec = new FFT(window);
	    
	    samplesPerShift = ((basr.getSampleRate() * windowShift) / 1000);		
	    int size = (viewNSamples / samplesPerShift);
	    spectrogramBuffer = new double[size][spec.getFrameSize()];
	     
	    for(int i = 0; i < size && spec.read(spectrogramBuffer[i]); i++) { }
	    repaintAndUpdateImage();
	}
\end{verbatim}

			\end{codeblock}	
		\end{columns}
\end{frame}

\begin{frame}
	\frametitle{Flexible Darstellungsm"oglichkeiten}
	\structure{}Anpassung des Spektrogramms an die Komponentengr"o"se\\[.3cm]
		\begin{itemize}
			\item Funktion zur Gr"o"sen"anderung des internen \structure{\texttt{BufferedImages}}
			\item Listener "uberwacht "Anderung an der Komponente
			\item Interface \structure{\texttt{java.awt.event.ComponentListener}}\\[.3cm]
		\end{itemize}
		\begin{columns}
			\column{0.05\linewidth} 
			\column{0.6\linewidth}
			\begin{codeblock}{\footnotesize Implementierung}
				\scriptsize
				\begin{verbatim}
	public void componentResized(ComponentEvent e) {
	     setImageDimension(getWidth(), getHeight());
	     repaintAndUpdateImage();
	}
\end{verbatim}

			\end{codeblock}
			\column{0.25\linewidth}
		\end{columns}
\end{frame}

\begin{frame}
	\frametitle{Normierung und Kontrastspreizung der Darstellung}
	\structure{Ziel: Verbesserung der Sichtbarkeit des Spektrogramms}\\[.3cm]
		\begin{itemize}
			\item Frequenzb"ander die stark ausgepr"agt sind sollen umso dunkler im Spektrogramm sein
			\item Abbildung von Werten aus der FFT auf Grauskala
			\item Formel zur Anpassung der Abstufungen im Wertebereich:\\[.3cm] $ pixelWert =  \dfrac{FFTWert}{Kontrast} ,  \quad Kontrast \geq FFTWert $
		\end{itemize}
\end{frame}

\begin{frame}
	\frametitle{Umsetzung der Kontrastanpassung}
		\begin{itemize}
			\item Kontrastdifferenzierung erh"oht bzw. verringert 'Lesbarkeit' des Spektrogramms\\[.3cm]
			\item Realisierung durch Einschr"ankung der Werteskala:\\[.3mm]
			\begin{itemize}
				\item kleiner Wertebereich: deutlichere Abstufungen zu benachbarten Graustufen
				\item gro"ser Wertebereich: kaum Unterschiede zu benachbarten Graustufen
			\end{itemize}
		\end{itemize}
		\begin{center}
			\includegraphics[width=0.9\linewidth]{\jpgdir/kontrastumfang.\jpg}
		\end{center}
\end{frame}

\begin{frame}
	\frametitle{Entkopplung von Anzeige und Berechnung}
	\structure{Ziel:} Einsparung von Rechenzeit bei Neuzeichnung der Komponente\\[.3cm]
		\begin{itemize}
			\item Zusatzinformationen bei jeder Mausbewegung in Komponente
			\item Information nicht ins Bild, sondern auf Komponente zeichnen
			\item Darstellung von Informationen \structure{ohne Neuberechnung} des Bildes 
			\item Transformation der Pufferdaten auf Bilddaten \structure{nur}, wenn sich am Spektrogramm etwas "andern soll: Kontrast, Farbe,...\\[.5cm]
			\item Entkopplung der Zeichenfunktionen: 
			\begin{itemize}
				\item \structure{\texttt{paint()}}
				\item \structure{\texttt{paintOverlay()}}
				\item \structure{\texttt{createSpectrogramImage}}
			\end{itemize}
		\end{itemize}
\end{frame}







%%%%%%%%%%%%%%%%%%%%%%%%%%%%%%%%%%%%%%%%%%%%%%%%%%%%%%%%%%%%%%%%%%%%%%%%%%%%%%%%

\section{Review und Ausblick}

\begin{frame}
	\frametitle{Was w"urden wir jetzt anders machen?}
	\structure{Think before you ink!}\\[.5cm]
	\begin{itemize}
		\item Mehr Vor"uberlegungen zu Projekt und Struktur
		\item Code-Leichen und 'hingehackte' Codest"ucke fr"uher "uber Board werfen ;-)
		\item Kommentare nicht erst am Ende
	\end{itemize}
\end{frame}


\begin{frame}
	\frametitle{Ausblick}
	\structure{Wenn wir noch mehr Zeit h"atten:}\\[.5cm]
	\begin{itemize}
		\item Weitere Verbesserungen der Speicherverwaltung
		\item Einzelne Features verbessern
			\begin{itemize}
				\item Vielf"altigere M"oglichkeiten zur Exportierung des Spektrogramms
				\item Weitere Ma"snahmen zur Synchronisation mit anderen Komponenten
			\end{itemize}
		\item Berechnungen parallelisieren\pause
		\item ...eine \structure{noch} sch"onere Demo-GUI schreiben :-)
	\end{itemize}
\end{frame}



%%%%%%%%%%%%%%%%%%%%%%%%%%%%%%%%%%%%%%%%%%%%%%%%%%%%%%%%%%%%%%%%%%%%%%%%%%%%%%%%

\section{Live Programmvorf"uhrung}

\begin{frame}
	\frametitle{Ende}
	\begin{center}
		\structure{\begin{Large}Vielen Dank f"ur die Aufmerksamkeit!\end{Large}}
	\end{center}
\end{frame}


\end{document}


